\documentclass[11pt]{article}
\usepackage{acronym}
\usepackage{titling}
\usepackage[small,compact]{titlesec}
\usepackage{enumitem}
\usepackage[margin=1.25in]{geometry}

\pagenumbering{gobble}

\setlist{nolistsep}

\title{MiniJava LLVM Front-end Proposal}
\author{Mitch Souders}
\date{April 25, 2014}
\begin{document}
\documentclass[11pt]{article}
\usepackage{acronym}
\usepackage{titling}
\usepackage[small,compact]{titlesec}
\usepackage{enumitem}
\usepackage[margin=1.25in]{geometry}

\pagenumbering{gobble}

\setlist{nolistsep}

\title{MiniJava LLVM Proposal}
\author{Mitch Souders and Mark Smith}
\date{April 25, 2014}
\begin{document}

\makeatletter
\def\maketitle{\par{\centerline{\huge\bfseries\@title}}\par\@author -- \@date}
\makeatother

\maketitle

\section*{Project Topic}
The primary goal of this project is to gain familiarity with \texttt{LLVM}. The proposal is to create an implementation for compiling mini-Java to the LLVM runtime.  The basis for the starting framework will be the mini-Java application used in CS321/322.  The compiler starter framework was provided by Professor Mark Jones.  The miniJava project currently includes options for running as an interpreter or compiling to x86 code.

The project proposal is to extend the back-end for these existing contructs to emit LLVM IL:

\begin{itemize} 
\item classes
\item class fields
\item class methods
\item variable declarations
\item this invocation
\item object access
\item statements and blocks
\item do-while loops
\item if-then constructs.  
\end{itemize}

\section*{Extensions}
The project proposal is to also extend the existing functionality by adding additional features to the miniJava compiler.  These would be added in the the following order:

\begin{itemize}
\item Garbage Collection
\item Dynamic Dispatch
\item Static Methods/Fields
\item Object Fields
\item Extensions to MiniJava (e.g. Interfaces)
\end{itemize}


\section*{Deliverables}
The project deliverables will be a summary of the work, any learning during implementation, the completed compiler as Java source code and unit tests indicating correct operations for language construct snippets.


\section*{Materials}
The following tools will be used to implement this project. Note: Some of these tools have alternates that could be used if the specified tool prove unsuitable (e.g. using \texttt{ANTLR} or \texttt{java\_cc} instead of \texttt{jcup}).

\begin{description}
\item[MiniJava]\footnote{http://www.cambridge.org/us/features/052182060X/}
	The desired front-end language to convert into \texttt{LLVM} IR. The grammar listed on the website will be assumed to be the canonical version of the language to be implemented.

\item[LLVM]\footnote{http://llvm.org/} The target compiler for the generated IR for MiniJava.

\item[jllvm]\footnote{https://code.google.com/p/jllvm}
Java bindings to \texttt{LLVM}, which allow direct interaction with \texttt{LLVM}.

\item[jcup]\footnote{http://www2.cs.tum.edu/projects/cup/}
Java parser generator necessary to build AST for MiniJava.

\end{description}

\end{document}

\makeatletter
\def\maketitle{\par{\centerline{\huge\bfseries\@title}}\par\@author -- \@date}
\makeatother

\maketitle

\section*{Project Topic}
The primary goal of this project is to gain familiarity with \texttt{LLVM} by creating a front-end for \texttt{MiniJava} for \texttt{LLVM}. 

\section*{Approach}
This will require writing an application in Java using the \texttt{jllvm} Java bindings to \texttt{LLVM} to produce the IR that is suitable for execution/compilation by \texttt{LLVM}. 

Note: The website for MiniJava does provide skeletons for many of the classes for MiniJava implementation with the logic left unimplemented.

The proposed flow is:
\begin{enumerate}
\item Lexing/Parsing of MiniJava with \texttt{jcup} creating an Abstract Syntax Tree.
\item Semantic Verification such as variable checks and type checking
\item Traversing AST using \texttt{jllvm} bindings to produce \texttt{LLVM} IR.
\end{enumerate}

\section*{Possible Extensions}
If this project does not meet the scope or is completed quickly, it may be necessary to add additional requirements. 

\begin{itemize}
\item Extensions to MiniJava (e.g. Interfaces)
\item Garbage Collection
\end{itemize}

\section*{Deliverables}
A working, demonstrable implementation of \texttt{MiniJava} using the \texttt{LLVM} front-end, along with all source code. A project progress report will be produced midway through the project. Additionally a project summary will also be provided, which will include implementation details, overcome issues, limitations of the \texttt{MiniJava} front-end and \texttt{LLVM}.

\section*{Materials}
The following tools will be used to implement this project. Note: Some of these tools have alternates that could be used if the specified tool prove unsuitable (e.g. using \texttt{ANTLR} or \texttt{java\_cc} instead of \texttt{jcup}).

\begin{description}
\item[MiniJava]\footnote{http://www.cambridge.org/us/features/052182060X/}
	The desired front-end language to convert into \texttt{LLVM} IR. The grammar listed on the website will be assumed to be the canonical version of the language to be implemented.

\item[LLVM]\footnote{http://llvm.org/} The target compiler for the generated IR for MiniJava.

\item[jllvm]\footnote{https://code.google.com/p/jllvm}
Java bindings to \texttt{LLVM}, which allow direct interaction with \texttt{LLVM}.

\item[jcup]\footnote{http://www2.cs.tum.edu/projects/cup/}
Java parser generator necessary to build AST for MiniJava.

\end{description}

\end{document}
